\documentclass[./main.tex]{subfiles}

\begin{document}

\chapter{Installation des outils}

\section{Installation de \LaTeX{}}

L'installation de \LaTeX{} est relativement simple, mais peut prendre un peu de temps et dépend de votre système d'exploitation.

\subsection{Sous Windows}

Sous Windows, le moyen le plus simple d'installer \LaTeX{} est d'utiliser Mik\TeX{}. Mik\TeX{} est une distribution \LaTeX{} pour Windows, qui inclut un gestionnaire de paquets pour installer facilement des extensions.

Pour installer Mik\TeX{}, rendez-vous sur le site officiel de Mik\TeX{} à l'adresse suivante : \url{https://miktex.org/download}.

Téléchargez le programme d'installation, et exécutez-le. Suivez les instructions à l'écran pour installer Mik\TeX{}.

\subsection{Sous macOS}

Sous macOS, le moyen le plus simple d'installer \LaTeX{} est d'utiliser Mac\TeX{}. Mac\TeX{} est une distribution \LaTeX{} pour macOS, qui inclut un gestionnaire de paquets pour installer facilement des extensions.

Pour installer Mac\TeX{}, rendez-vous sur le site officiel de Mac\TeX{} à l'adresse suivante : \url{https://www.tug.org/mactex/}.

Téléchargez le programme d'installation, et exécutez-le. Suivez les instructions à l'écran pour installer Mac\TeX{}.

\subsection{Sous Linux}

Sous Linux, le moyen le plus simple d'installer \LaTeX{} est d'utiliser TeX Live. TeX Live est une distribution \LaTeX{} pour Linux, qui inclut un gestionnaire de paquets pour installer facilement des extensions.

Pour installer TeX Live, ouvrez un terminal, et exécutez la commande suivante :
"sudo apt-get install texlive-full" pour les distributions basées sur Debian, ou "sudo dnf install texlive-scheme-full" pour les distributions basées sur Fedora.

\section{Installation d'un éditeur de texte}

Pour écrire des documents \LaTeX{}, vous aurez besoin d'un éditeur de texte. Il existe de nombreux éditeurs de texte adaptés à \LaTeX{}, nous utiliserons dans ce document Visual Studio Code.

Visual Studio Code est un éditeur de texte gratuit, open-source, et multi-plateforme, développé par Microsoft.

Pour installer Visual Studio Code, rendez-vous sur le site officiel de Visual Studio Code à l'adresse suivante : \url{https://code.visualstudio.com/}. Téléchargez le programme d'installation, et exécutez-le. Suivez les instructions à l'écran pour installer Visual Studio Code.

\section{Installation de git}

Pour pouvoir enregistrez vos documents \LaTeX{}, je vous recommande d'utiliser git. Git est un logiciel de gestion de versions, qui permet de suivre l'évolution de vos documents, et de les partager avec d'autres personnes, ce qui est particulièrement utile pour travailler en équipe.

Pour ma part, j'utilise git à travers le module git intégré à Visual Studio Code. Puisque cela ne nécessite pas d'installation supplémentaire, je ne détaillerai pas ici l'installation de git, ni son utilisation. Si vous souhaitez en savoir plus sur git, je vous recommande de consulter la documentation officielle de git à l'adresse suivante : \url{https://git-scm.com/}.

\section{Installation de Lua\LaTeX{}}

Pour installer Lua\LaTeX{}, vous n'avez rien à faire de particulier, car Lua\LaTeX{} est inclus dans les distributions \LaTeX{} modernes, telles que Mik\TeX{}, Mac\TeX{}, ou TeX Live.

\end{document}