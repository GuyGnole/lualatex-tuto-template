\documentclass[../main.tex]{subfiles}

\begin{document}

\chapter{Introduction}

\section{Qu'est-ce que \LaTeX{} ?}

\LaTeX{} est un système de composition de documents créé par Leslie Lamport en 1983, basé sur le logiciel TeX créé par Donald Knuth en 1978. \LaTeX{} est un logiciel libre disponible sur la plupart des systèmes d'exploitation.

Il est particulièrement adapté à la rédaction de documents scientifiques, techniques, ou académiques, mais peut être utilisé pour tout type de document. Il est particulièrement apprécié pour la qualité typographique de ses documents et pour sa capacité à gérer de très longs documents.

\section{Pourquoi utiliser Lua\LaTeX{} ?}

Lua\LaTeX{} est une version de \LaTeX{} qui permet d'utiliser le langage de programmation Lua pour étendre les fonctionnalités de \LaTeX{}. Lua\LaTeX{} est particulièrement adapté pour la création de documents multilingues ou pour la création de documents nécessitant des calculs ou des traitements de données.

\section{Comment utiliser \LaTeX{} ?}

Si vous voulez éviter de prendre du temps pour installer les outils nécessaires à l'utilisation de \LaTeX{}, vous pouvez utiliser un éditeur en ligne. Il en existe plusieurs, mais Overleaf est l'un des plus populaires. Overleaf est un éditeur en ligne qui permet de créer des documents \LaTeX{} sans avoir à installer quoi que ce soit sur votre ordinateur. Il est particulièrement adapté pour les débutants car il propose de nombreux modèles de documents et permet de compiler les documents en temps réel.

Dans ce document, nous allons installer ensemble les outils nécessaires directement sur votre ordinateur.

\end{document}
